\documentclass[10pt,twocolumn]{article}
\title{Distribution of Unmodified Binaries}
\author{Nathan Beckmann}

\usepackage{verbatim}

\begin{document}

\maketitle

\begin{abstract}
  We'd like to be able to distribute unmodified application binaries
  across multiple processes. This is complicated by constraining the
  main thread to a single process, and being able to spawn threads in
  other processes from this ``main process''.
\end{abstract}

\section{Goals}
\label{sec:goals}

Currently, we are distributing binaries by changing the user
application to be MPI-aware. The application is modified to spawn the
correct threads for its process ID and only do ``main stuff'' in
process zero. What we would like to be able to do is:
\begin{verbatim}
$ mpirun -np X <pin on binary>
\end{verbatim}
And have things ``just work''.

This is complicated because, without modifications, the application
will just run \texttt{X} copies of itself in parallel. We need to
somehow let one process run the main thread, and have the others go
into ``stand-by'', waiting for thread spawn requests.

\section{Approach}
\label{sec:approach}

We use PIN's mechanisms for replacing and calling application
functions to achieve our goals.

\subsection{Thread creation}
\label{sec:thread-creation}

\subsubsection{Main process}
\label{sec:main-process}

In the main process, we replace all \texttt{pthread} thread-related
calls with simulator functions. In effect, we need to re-implement a
subset of \texttt{pthread}s to interact with threads distributed
across the processes.

We replace \texttt{pthread\_create} with a function that stripes the
threads across the processes. When a thread is to be created remotely,
it sends a message to that process with the parameters passed to
\texttt{pthread\_create} (function address and argument), and that
process spawns the thread. (See next section.) When the thread should
live locally, then we use \texttt{PIN\_CallApplicationFunction} to let
the \texttt{pthread\_create} call fall through.

\subsubsection{Other processes}
\label{sec:other-processes}

In the remaining processes, we \texttt{RTN\_Replace} the
\texttt{main()} function of the application with a simulator
function. This function simply blocks waiting for thread creation
requests from the main process. When a request comes, then it extracts
the parameters to \texttt{pthread\_create} that are in the message and
uses \texttt{PIN\_CallApplicationFunction} to invoke
\texttt{pthread\_create}, just as described above.

\subsection{Thread joining}
\label{sec:joining}

We also must re-implement \texttt{pthread\_join} so that it works
appropriately with remote threads. In particular, the value of
\texttt{pthread\_create} must change to some custom identifier (the
\texttt{commID} of the new thread?).

Then, the MCP must keep a table of thread states for remote
threads. In order for this table to be properly updated, when a thread
is created, we must send the MCP a message when the thread exits. This
can be done with \texttt{PIN\_AddThreadFini}, which must be called when
the thread is created.

\subsection{Alternatives}
\label{sec:alternatives}

It might be simpler to remove the main process as a special case ---
that is, don't have any spawned threads live in the main process. This
would make this process extremely light-weight, and its only job would
be to house the MCP and deal with the remote threads. Or, a hybrid
approach is to give the main process its own thread that waits for
thread spawning requests. This hybrid is what is described in the next
section.

\section{Implementation}
\label{sec:implementation}

\subsection{Pseudocode}
\label{sec:pseudocode}

This section describes the important events that take place in the
system and the actions that are done in response.

\subsubsection{Application start-up (in pin tool):}
\begin{enumerate}
\item Get MPI ID.
\item Replace \texttt{main} with \texttt{SimMain}.
\item Replace \texttt{pthread\_create} with \texttt{SimThreadCreate}.
\item Replace \texttt{pthread\_join} with \texttt{SimThreadJoin}.
\item Do normal instrumentation.
\item Start application.
\end{enumerate}

\subsubsection{\texttt{SimMain}:}
\begin{enumerate}
\item Spawn a PIN thread with \texttt{SimThreadSpawnLoop}.
\item If we are process zero, call \texttt{main} via
  \texttt{PIN\_CallApplicationFunction}.
\end{enumerate}

\subsubsection{\texttt{SimThreadCreate}:}
\begin{enumerate}
\item Compute which process the thread should live on, based on which
  process currently has the least active threads. (Use MCP internal
  thread state table.)
\item Send a message to the process's thread spawner with the function
  address and data parameter.
\item Update MCP internal state to make thread ``active''.
\item Return identifier for thread (its \texttt{commID}).
\end{enumerate}

\subsubsection{\texttt{SimThreadJoin}:}
\begin{enumerate}
\item Query MCP state for requested thread.
\item \label{check} If ``inactive'', return.
\item If ``active'', wait on condition variable until thread state is
  updated.
\item Go to step \ref{check}.
\end{enumerate}

\subsubsection{\texttt{SimThreadSpawnLoop}:}
\begin{enumerate}
\item Initialize phys trans.
\item \label{wait} Wait for thread creation message.
\item Using \texttt{PIN\_CallApplicationFunction}, call
  \texttt{pthread\_create} with the parameters.
\item Call \texttt{PIN\_AddThreadFini} to add
  \texttt{SimThreadExitCallback} with the new thread's ID. (What is
  the exact mechanism for doing this? Does the value that
  \texttt{pthread\_create} returns work?)
\item Go to step \ref{wait}.
\end{enumerate}

\subsubsection{\texttt{SimThreadExitCallback}:}
\begin{enumerate}
\item Send message to MCP with our thread ID notifying it that we have
  exited.
\end{enumerate}

\subsubsection{MCP receives thread exit message:}
\begin{enumerate}
\item Update internal thread state for the thread as ``inactive''.
\item Signal main thread that state has been updated.
\end{enumerate}

\subsection{Comments}
\label{sec:comments}

\begin{itemize}
\item The physical transport layer might need to be modified so that
  it is standalone and can be used by the ``other processes'' without
  a core, network, etc..
\item \texttt{main()} running on ``other processes'' needs an MPI ID,
  but there is no core associated with them. They are kind of like
  ``mini-MCPs'', and so a resurrection of the special MPI physical
  transport code might be in order.
\item We assume that a function address that is valid in one process
  will be valid in another. Virtual memory should make this so. But if
  we run into problems, then we will have to build a map from function
  names to function addresses and pass the function name from one
  process to another. Because we are abstracting the shared memory,
  the data parameter should remain valid.
\item Based on this pseudocode, the MCP state table must contain:
  thread ID, process, state.
\end{itemize}

\end{document}
